%SECCIÓN 3. PREGUNTAS INVESTIGACION


\section{Preguntas de investigación}
Las preguntas de investigación juegan un papel importante para el desarrollo de una investigacion de esta indole, ya que a través de ellas se logra una mejor interpretación  y definición del problema.  Las preguntas de investigación se clasifican en varios tipos de acuerdo al análisis que se desea lograr y en este caso se van a desarrollar las siguientes:
 \subsection{Preguntas de carácter descriptivo}
 Cuando se responde las preguntas de carácter descriptivo ya se puede identificar y conocer las características iniciales del conjunto de datos. Las preguntas de carácter descriptivo son:
  \begin{itemize}
	 \item ¿La elección de realizar estudios profesionales en la universidad publica se da por que es una opción para reducción de costos?
	 \item ¿La elección de realizar estudios profesionales en la universidad publica se da por obtener reconocimiento académico de la misma?
  	
  \end{itemize}
  \subsection{Preguntas de carácter exploratorio}
   Las preguntas de carácter exploratorio consisten en la búsqueda de patrones o relaciones que soporten una pregunta de investigación.
   	  \begin{itemize}
  	  	\item ¿Existe algún tipo de relación entre el estrato socio económico y el desarrollo intelectual?
  	  	\item ¿Los estudiantes falsean el estrato social real, para obtener beneficios económicos en el valor de la matricula?
  	  	\item ¿La elección de estudiar en la Universidad de los Llanos, se realiza para reducir costos?
  	  	\item ¿La elección de estudiar en la Universidad de los Llanos, se realiza para obtener reconocimiento académico de la misma?
  	  \end{itemize}

  \subsection{Preguntas de carácter inferencial}
   Las preguntas de carácter inferencial consisten en el planteamiento de una hipótesis que podría ser resuelta con el análisis respectivo de la información
  \begin{itemize}
   \item    \item ¿Existe algún tipo de relación entre el estrato socio económico y el desarrollo intelectual?
  \end{itemize}
  \subsection{Preguntas de carácter predictivo}
   Las preguntas de carácter predictivo permiten analizar el comportamiento de la información a través del tiempo, con el objetivo de descubrir, proyectar, o realizar hipótesis sobre estados futuros.
	\begin{itemize}
		\item ¿Por definir?
	\end{itemize}