%SECCIÓN 3. CONCLUSIONES

\section{CONCLUSIONES}

\begin{itemize}
 	
 	\item Los estudiantes que afirman vivir en estrato socioeconómico 1 tienen un 62\% de influencia en los promedios de carrera que superan 3.5, en contraste a lo que sucede en el estrato 2, puesto que las diferencias entre los promedios se mantienen en igual número. En el estrato tres las notas que superan el tres con cinco es de 53\% lo que permite afirmar que el estrato  en que vive la persona si influye de manera determinante en la obtención de mejores promedios de carrera.
 	
 	\item El estado económico es un aspecto muy importante en la vida de toda persona puesto que se han establecido muchas diferencias entre el ritmo de vida de una persona de estrato alto en comparación con aquella de estrato bajo, este aspecto esta íntimamente ligado al bienestar y la necesidad de obtener dinero para subsanar los gastos de manutención básicos, esta estrecha relación entre dinero y bienestar deja en evidencia aspectos que afectan el desarrollo intelectual y por lo tanto académico del estudiante.
 	
 	\item Al analizar a nivel socioeconómico los factores asociados con la deserción, se observa que a lo largo del período estudiado éstos son un determinante constante respecto de las demás causas, y si se tiene en cuenta que el estrato socioeconómico de la población estudiantil en los programas de pregrado de la universidad, se concentra en los estratos 1, 2 y 3 con un 97.0\%, evidencia la  capacidad económica de los hogares de los estudiantes,
 	
 	\item El estrato socioeconómico incide en el rendimiento académico de los 5000 estudiantes de la Universidad de los Llanos, hemos establecido que este aspecto esta íntimamente ligado al bienestar y la necesidad de obtener dinero para subsanar los gastos de manutención básicos, esta estrecha relación entre dinero y bienestar deja en evidencia aspectos que afectan el desarrollo intelectual y por lo tanto académico del estudiante. 
\end{itemize}
