	

%Iniciar Abstract
\begin{abstract}
It has been established that economic stratum is an aspect that determines the level of access to products and services that generate welfare in the life of a person, for this reason a person of high stratum compared with that of the lower stratum, it is difficult to obtaining welfare, beyond what can be termed non-core expenses.
\bigskip
The socioeconomic stratum is related to the amount of money that is accessed by a person to meet their material needs, but is there any relationship between socioeconomic strata and intellectual development? Is the choice to carry out professional studies at university public is given more to be a better option in cost reduction or rather academic recognition obtainable from it?
\end{abstract}