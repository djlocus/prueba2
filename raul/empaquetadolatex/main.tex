%Tipo de Documento [Conferencia]

\documentclass[conference]{IEEEtran}\usepackage[]{graphicx}\usepackage[]{color}
%% maxwidth is the original width if it is less than linewidth
%% otherwise use linewidth (to make sure the graphics do not exceed the margin)
\makeatletter
\def\maxwidth{ %
  \ifdim\Gin@nat@width>\linewidth
    \linewidth
  \else
    \Gin@nat@width
  \fi
}
\makeatother

\definecolor{fgcolor}{rgb}{0.345, 0.345, 0.345}
\newcommand{\hlnum}[1]{\textcolor[rgb]{0.686,0.059,0.569}{#1}}%
\newcommand{\hlstr}[1]{\textcolor[rgb]{0.192,0.494,0.8}{#1}}%
\newcommand{\hlcom}[1]{\textcolor[rgb]{0.678,0.584,0.686}{\textit{#1}}}%
\newcommand{\hlopt}[1]{\textcolor[rgb]{0,0,0}{#1}}%
\newcommand{\hlstd}[1]{\textcolor[rgb]{0.345,0.345,0.345}{#1}}%
\newcommand{\hlkwa}[1]{\textcolor[rgb]{0.161,0.373,0.58}{\textbf{#1}}}%
\newcommand{\hlkwb}[1]{\textcolor[rgb]{0.69,0.353,0.396}{#1}}%
\newcommand{\hlkwc}[1]{\textcolor[rgb]{0.333,0.667,0.333}{#1}}%
\newcommand{\hlkwd}[1]{\textcolor[rgb]{0.737,0.353,0.396}{\textbf{#1}}}%

\usepackage{framed}
\makeatletter
\newenvironment{kframe}{%
 \def\at@end@of@kframe{}%
 \ifinner\ifhmode%
  \def\at@end@of@kframe{\end{minipage}}%
  \begin{minipage}{\columnwidth}%
 \fi\fi%
 \def\FrameCommand##1{\hskip\@totalleftmargin \hskip-\fboxsep
 \colorbox{shadecolor}{##1}\hskip-\fboxsep
     % There is no \\@totalrightmargin, so:
     \hskip-\linewidth \hskip-\@totalleftmargin \hskip\columnwidth}%
 \MakeFramed {\advance\hsize-\width
   \@totalleftmargin\z@ \linewidth\hsize
   \@setminipage}}%
 {\par\unskip\endMakeFramed%
 \at@end@of@kframe}
\makeatother

\definecolor{shadecolor}{rgb}{.97, .97, .97}
\definecolor{messagecolor}{rgb}{0, 0, 0}
\definecolor{warningcolor}{rgb}{1, 0, 1}
\definecolor{errorcolor}{rgb}{1, 0, 0}
\newenvironment{knitrout}{}{} % an empty environment to be redefined in TeX

\usepackage{alltt}

%BIBLIOTECAS

% Este paquete se utiliza para generar texto o graficas de relleno.
%\usepackage{blindtext, graphicx}
%Biblioteca para graficas
\usepackage{graphicx}
%Biblioteca para lectura de caracteres ortográficos (tildes..etc. ) 
\usepackage[utf8]{inputenc}
%Biblioteca para graficos vectrizados.svg 
\usepackage{svg}
\usepackage{enumerate}
%Biblioteca para enumerar figuras tablas.. etc en español 
\usepackage[english]{babel}
\usepackage{hyperref}


%INICIO DEL DOCUMENTO
\IfFileExists{upquote.sty}{\usepackage{upquote}}{}
\begin{document}


% TITULO DEL PAPER
\title{Data analysis of the Online Retails DataSet}


% NOMBRE DE LOS AUTORES
\author{
 \IEEEauthorblockN{Raul Alejandro Buitrago Castellanos}
 \IEEEauthorblockA{Ingeniero de Sistemas\\ 
  Universidad Distrital Francisco José de Caldas\\
  Bogotá D.C., Colombia\\
  Email: raulhabits@gmail.com}
}

%TITULO
\maketitle

	

%Iniciar Abstract
\begin{abstract}
It has been established that economic stratum is an aspect that determines the level of access to products and services that generate welfare in the life of a person, for this reason a person of high stratum compared with that of the lower stratum, it is difficult to obtaining welfare, beyond what can be termed non-core expenses.
\bigskip
The socioeconomic stratum is related to the amount of money that is accessed by a person to meet their material needs, but is there any relationship between socioeconomic strata and intellectual development? Is the choice to carry out professional studies at university public is given more to be a better option in cost reduction or rather academic recognition obtainable from it?
\end{abstract} 

%Iniciar Palabras Clave Formato IEEE
\begin{IEEEkeywords}
	Dataset, Data Mining, Big Data Analytics, Business intelligence.
\end{IEEEkeywords}

%SECCIÓN 1. INTRODUCCIÓN 
\section{Introducción}
	El análisis de Desminado en Colombia es tema relevante para la actualidad ya que podría mostrar como al acercarse un acuerdo de Paz se ha aumentado o disminuido el Desminado en Colombia así como poder determinar cuales son los departamentos que mas se han visto afectados entre otros.

\section{Objectives}
 Make a data analysis to the dataset using statistic techniques to identify the influence between the variables establishing behaviors or patterns to understanding and modeling the Online Retails information, using bigdata and data mining techniques.
\section{Theoretical Framework}
To a complete understanding of this document the reader must take some minutes to check the following terms.
 \subsection{Statistics}
 Is the studying of random phenomena, to obtain some conclusions from a test dataset.
 \subsection{Dataset}
 Is part of the statistic process, because it's a collection of records used to describe the population.
 \subsection{Data mining}
 Is the use of statistic tools, and computing science approach to find patterns in a dataset.
 \subsection{Machine learning}
 Is a set of techniques used in the data mining to establish relationships and predict the behavior of a dataset using some IA algorithms.
 \subsection{Big Data Analytics}
 Is the combination between business intelligence and analytics techniques, those are commonly used in data mining and statistical analysis. Some of the techniques mentioned are the K-Means, Naive Bayes, K-Nearest neighbor, clustering, and regression.
 

\section{State of the art} 
 There are several ways to use the data science applied in marketing, and the use of data mining techniques, in this case for this dataset some authors made a representative work based in clustering and decision trees, obtaining good results establishing some relationships between some fields obtaining patterns.\\
 Some of the comments that other authors provide us in their researches are that the data mining process takes a long time, because those tasks are too complex
 \begin{itemize}
  \item{Data preparation} 
  \item{Model interpretation}
  \item{Evaluation}
 \end{itemize}
 And another considerations are related with the correlation, because correlation in some cases doesn't implies causation.
 
%SECCIÓN 3. METODOLOGIA
\section{METODOLOGÍA}

 Para el desarrollo de este documento se utilizará como marco de referencia la metodología para minería de datos y fundamentos para análisis de datos [1].
 
  \subsection{Comprensión del negocio}
  
   \begin{itemize}
   	
   	\item \textbf{Identificar un problema}: El Gobierno Colombiano ha implementado un sistema de Seguridad Social el cual está compuesto por el régimen contributivo y subsidiado. En cuando al  régimen contributivo, se deben afiliar las personas que tienen una vinculación laboral, es decir, con capacidad de pago como los trabajadores formales e independientes, los pensionados y sus familias.  Debido al crecimiento poblacional Colombiano en los últimos años [2] es importante conocer el comportamiento las contribuciones a este sistema, y por consiguiente se hace necesario poder analizar el monto de los aportes al sistema de seguridad social según la edad del cotizante. Para ello, utilizaremos los datos abiertos publicados por el DANE.\\ 
   	
   \end{itemize} 
   
    \subsection{Preguntas de investigación}
    La pregunta de investigación que se desarrollará en este artículo estará enmarcada en los siguientes ámbitos:
    \begin{itemize}
    	\item Descriptivas 
    	\item Exploratorias
    	\item Inferenciales
    \end{itemize}
    
    Pregunta de Investigación: 
    \begin{itemize}
    	\item ¿El monto de los aportes de los afiliados al sistema de seguridad social colombiano aumenta con relación a la edad del cotizante?
    \end{itemize}
  
  \subsection{Comprensión de los Datos}
 
   \begin{itemize}
   \item \textbf{Recopilación inicial de datos}: Los datos obtenidos fueron exportados desde la página del Departamento Nacional de Estadísticas [2],  con el objetivo de analizar el monto de los aportes al sistema de seguridad social considerando la variable edad del cotizante.\\
   
   \item \textbf{Descripción de los datos: Variables del DataSet}: El conjunto de datos fueron descargados de la página del DANE en formato csv, formato separado por comas,, el contenía una población total de 10.000 registros.  el conjunto de datos está compuesto de los siguientes campos: 
   \\ 	 
   Género. Hombre - Mujer\\
   Año de Nacimiento\\
   Años cumplidos\\ 
   Valor mensual de la cotización a salud?\\
     
\end{itemize} 
  	
	
  \subsection{Preparación de los Datos}
  Siguiendo la metodología para minería de datos, se adelantarán las siguientes actividades 
  \begin{itemize}
  	\item Selección de los datos
  	\item Limpieza de datos
	\item Construcción de datos
    \item Integración de datos
	\item Formateo de datos
   \end{itemize}  
   Dado que la política de datos abiertos exigen que la disponibilidad de los datos al público, estén ya estructurados con antelación.  Dichas actividades no se llevan a cabo dentro de este artículo.

  \subsection{Análisis exploratorio}
Durante el análisis de la información de DANE se utilizan los siguientes aspectos para el desarrollo de este artículo:

  \begin{itemize}
  
   \item Frecuencia relativa    
   \item Medidas de tendencia central[4] 
	   \begin{itemize}
		 	\item Media. 
		 	\item Moda. 
		 	\item Mediana.
		 \end{itemize}
 %15\
 
  \item Medidas de Dispersión[4] 
  \begin{itemize}
 
  \item Varianza.  
 
 
     \item Desviaci\'on est\'andar .
\end{itemize} 

\end{itemize}  




%SECCIÓN 3. PREGUNTAS INVESTIGACION
\section{Information Recognizing}
In this section will be described the data domain, the dataset, and the variables recognizing, and the SMART objectives.

 \subsection{The domain of the data.}
 One of the common uses of the business intelligence is oriented to the marketing, because all the organizations needs to know the current state of them, the history of their behavior, the capacity to satisfy the client requests and to increase their competitive.\\
 In this data analysis the dataset used is provided by  \href{http://archive.ics.uci.edu/ml/datasets/Online+Retail}{UCI on the official web site} and it contains some records related with the transactions of a store through a time period.
 \subsection{The Dataset}
  Like the UCI web page says, the dataset contains all the transactions occurring between 01/12/2010 and 09/12/2011 for a UK-based and registered non-store online retail.The company mainly sells unique all-occasion gifts. Many customers of the company are wholesalers, it's made off 541909 records of online retails and is composed by a the next fields.
 \begin{itemize}
  \item{InvoiceNo.} Invoice number. Nominal, a 6-digit integral number uniquely assigned to each transaction. If this code starts with letter 'c', it indicates a cancellation. 
  \item{StockCode.} Product (item) code. Nominal, a 5-digit integral number uniquely assigned to each distinct product. 
  \item{Description.} Product (item) name. Nominal. 
  \item{Quantity.} The quantities of each product (item) per transaction. Numeric.	
  \item{InvoiceDate.} Invoice Date and time. Numeric, the day and time when each transaction was generated. 
  \item{UnitPrice.} Unit price. Numeric, Product price per unit in sterling. 
  \item{CustomerID.} Customer number. Nominal, a 5-digit integral number uniquely assigned to each customer. 
  \item{Country.} Country name. Nominal, the name of the country where each customer resides.
 \end{itemize}
 \subsection{Smart Objectives}
 Check the behavior of the online retails from the past, to establish patterns, tendencies, and behaviors of the market.\\
 This objective must be the of high quality because the stakeholder may use it to take decisions.






  

  

  

  

  

  

  
    


%SECCIÓN 3. PREGUNTAS INVESTIGACION
\section{Research Questions}
A correct investigation begins with the right questions, then an optimal definition of what else will be the proposal of the research. The following questions are used in this research, these questions must be classified according with the kind of it.

  \subsection{Descriptive.}
  A descriptive question is used to identify and know the characteristics of the dataset.
  \begin{itemize}
   \item What's the min date of the measurement?
\begin{knitrout}
\definecolor{shadecolor}{rgb}{0.969, 0.969, 0.969}\color{fgcolor}\begin{kframe}
\begin{verbatim}
## [1] "2010-12-01"
\end{verbatim}
\end{kframe}
\end{knitrout}
   \item What's the max date of the measurement?
\begin{knitrout}
\definecolor{shadecolor}{rgb}{0.969, 0.969, 0.969}\color{fgcolor}\begin{kframe}
\begin{verbatim}
## [1] "2011-12-09"
\end{verbatim}
\end{kframe}
\end{knitrout}
   \item What's the product quantity sold in the measurement?
\begin{knitrout}
\definecolor{shadecolor}{rgb}{0.969, 0.969, 0.969}\color{fgcolor}\begin{kframe}
\begin{verbatim}
## [1] 5660981
\end{verbatim}
\end{kframe}
\end{knitrout}
   \item What's the product quantity returned in the measurement?
\begin{knitrout}
\definecolor{shadecolor}{rgb}{0.969, 0.969, 0.969}\color{fgcolor}\begin{kframe}
\begin{verbatim}
## [1] 484531
\end{verbatim}
\end{kframe}
\end{knitrout}
  \end{itemize}
  \subsection{Exploratory.}
   An exploratory question consists in the searching of patterns or relations to support an investigation question.
  \begin{itemize}
   \item Which month had major product sales and devolutions?
\begin{knitrout}
\definecolor{shadecolor}{rgb}{0.969, 0.969, 0.969}\color{fgcolor}
\includegraphics[width=\maxwidth]{figure/monthMajorSalesDevolutionsQuantity-1} 

\end{knitrout}

% latex table generated in R 3.3.0 by xtable 1.8-2 package
% Sat May 28 14:51:01 2016
\begin{table}[ht]
\centering
\begin{tabular}{rrlrr}
  \hline
 & Index & YearMonth & PositiveQuantity & NegativeQuantity \\ 
  \hline
1 &   1 & 2010 - 12 & 362316.00 & 20088.00 \\ 
  42482 &   2 & 2011 - 1 & 397716.00 & 88750.00 \\ 
  77629 &   3 & 2011 - 2 & 286695.00 & 8706.00 \\ 
  105336 &   4 & 2011 - 3 & 384950.00 & 33078.00 \\ 
  142084 &   5 & 2011 - 4 & 312176.00 & 23078.00 \\ 
  172000 &   6 & 2011 - 5 & 399425.00 & 19034.00 \\ 
  209030 &   7 & 2011 - 6 & 394337.00 & 52714.00 \\ 
  245904 &   8 & 2011 - 7 & 407539.00 & 16423.00 \\ 
  285422 &   9 & 2011 - 8 & 425016.00 & 18817.00 \\ 
  320706 &  10 & 2011 - 9 & 575416.00 & 25599.00 \\ 
  370932 &  11 & 2011 - 10 & 628745.00 & 58213.00 \\ 
  431674 &  12 & 2011 - 11 & 771598.00 & 31312.00 \\ 
  516385 &  13 & 2011 - 12 & 315052.00 & 88719.00 \\ 
   \hline
\end{tabular}
\caption{Dataset of product sales and devolution} 
\end{table}

   \item Which country bought the major products quantity in January of 2011?

% latex table generated in R 3.3.0 by xtable 1.8-2 package
% Sat May 28 14:51:01 2016
\begin{table}[ht]
\centering
\begin{tabular}{rlrrr}
  \hline
 & Country & PositiveQuantity & NegativeQuantity & Index \\ 
  \hline
44153 & Bahrain & 0.00 & 54.00 &   1 \\ 
  47164 & Japan & 0.00 & 45.00 &   2 \\ 
  50792 & Israel & 100.00 & 0.00 &   3 \\ 
  54381 & Cyprus & 144.00 & 0.00 &   4 \\ 
  72851 & Channel Islands & 259.00 & 4.00 &   5 \\ 
  58992 & Poland & 288.00 & 0.00 &   6 \\ 
  72247 & Iceland & 315.00 & 0.00 &   7 \\ 
  72986 & Lebanon & 386.00 & 0.00 &   8 \\ 
  69008 & Greece & 526.00 & 0.00 &   9 \\ 
  66153 & Finland & 765.00 & 0.00 &  10 \\ 
  43858 & Belgium & 792.00 & 9.00 &  11 \\ 
  70759 & Singapore & 1091.00 & 0.00 &  12 \\ 
  43780 & Italy & 1121.00 & 25.00 &  13 \\ 
  69624 & Hong Kong & 1121.00 & 0.00 &  14 \\ 
  43781 & Portugal & 2094.00 & 16.00 &  15 \\ 
  62999 & Switzerland & 2993.00 & 0.00 &  16 \\ 
  43420 & Sweden & 3097.00 & 1.00 &  17 \\ 
  45623 & Spain & 3845.00 & 8.00 &  18 \\ 
  45512 & Australia & 5644.00 & 0.00 &  19 \\ 
  44295 & EIRE & 8794.00 & 106.00 &  20 \\ 
  44206 & Germany & 9077.00 & 171.00 &  21 \\ 
  43857 & France & 9199.00 & 44.00 &  22 \\ 
  57394 & Netherlands & 20417.00 & 0.00 &  23 \\ 
  42482 & United Kingdom & 325648.00 & 88267.00 &  24 \\ 
   \hline
\end{tabular}
\caption{Information of buys in January of 2011} 
\end{table}

   
   \item Which country had major sales and devolutions of products?
% latex table generated in R 3.3.0 by xtable 1.8-2 package
% Sat May 28 14:51:01 2016
\begin{table}[ht]
\centering
\begin{tabular}{rrlrr}
  \hline
 & Index & Country & PositiveQuantity & NegativeQuantity \\ 
  \hline
100811 &   1 & Saudi Arabia & 80.00 & 5.00 \\ 
  38314 &   2 & Bahrain & 314.00 & 54.00 \\ 
  395473 &   3 & RSA & 352.00 & 0.00 \\ 
  157300 &   4 & Brazil & 356.00 & 0.00 \\ 
  72986 &   5 & Lebanon & 386.00 & 0.00 \\ 
  168150 &   6 & European Community & 499.00 & 2.00 \\ 
  7987 &   7 & Lithuania & 652.00 & 0.00 \\ 
  103599 &   8 & Czech Republic & 671.00 & 79.00 \\ 
  217685 &   9 & Malta & 970.00 & 26.00 \\ 
  89571 &  10 & United Arab Emirates & 982.00 & 0.00 \\ 
  69008 &  11 & Greece & 1557.00 & 1.00 \\ 
  14939 &  12 & Iceland & 2458.00 & 0.00 \\ 
  164465 &  13 & USA & 2458.00 & 1424.00 \\ 
  119192 &  14 & Canada & 2763.00 & 0.00 \\ 
  152713 &  15 & Unspecified & 3300.00 & 0.00 \\ 
  6609 &  16 & Poland & 3684.00 & 31.00 \\ 
  31983 &  17 & Israel & 4409.00 & 56.00 \\ 
  69624 &  18 & Hong Kong & 4773.00 & 4.00 \\ 
  31465 &  19 & Austria & 4881.00 & 54.00 \\ 
  70759 &  20 & Singapore & 5241.00 & 7.00 \\ 
  29733 &  21 & Cyprus & 6361.00 & 44.00 \\ 
  7215 &  22 & Italy & 8112.00 & 113.00 \\ 
  20018 &  23 & Denmark & 8235.00 & 47.00 \\ 
  20001 &  24 & Channel Islands & 9491.00 & 12.00 \\ 
  34084 &  25 & Finland & 10704.00 & 38.00 \\ 
  7135 &  26 & Portugal & 16258.00 & 78.00 \\ 
  1237 &  27 & Norway & 19338.00 & 91.00 \\ 
  7280 &  28 & Belgium & 23237.00 & 85.00 \\ 
  9784 &  29 & Japan & 26016.00 & 798.00 \\ 
  6422 &  30 & Spain & 27951.00 & 1127.00 \\ 
  5321 &  31 & Switzerland & 30630.00 & 305.00 \\ 
  30079 &  32 & Sweden & 36083.00 & 446.00 \\ 
  198 &  33 & Australia & 84209.00 & 556.00 \\ 
  27 &  34 & France & 112104.00 & 1624.00 \\ 
  1110 &  35 & Germany & 119263.00 & 1815.00 \\ 
  1405 &  36 & EIRE & 147447.00 & 4810.00 \\ 
  386 &  37 & Netherlands & 200937.00 & 809.00 \\ 
  1 &  38 & United Kingdom & 4733819.00 & 469990.00 \\ 
   \hline
\end{tabular}
\caption{Global information about the countries consumption} 
\end{table}

   
  \end{itemize}
  \subsection{Inferential.}
   An inferential question consists in the creation of an hypothesis to be solved analyzing the information.
  \begin{itemize}
   \item Was France the country with major sales in January of 2011?\\
   The Table III suggest that United Kingdom had the major sales in January of 2011
  \end{itemize}


%SECCIÓN 3. PREGUNTAS INVESTIGACION
\section{Exploratory Analysis}
In this section was made an analysis in some scenarios to check the information related with the "frequencies" and the "central tendency measures"\\

 The following information contains the resume of the exploratory analysis of the variables through the measure period
% latex table generated in R 3.3.0 by xtable 1.8-2 package
% Sat May 28 14:51:01 2016
\begin{table}[ht]
\centering
\begin{tabular}{rlrrrrr}
  \hline
 & Variable & mean & min & max & var & sd \\ 
  \hline
1 & Year & 2010.92 & 2010.00 & 2011.00 & 0.07 & 0.27 \\ 
  2 & Month & 7.55 & 1.00 & 12.00 & 12.31 & 3.51 \\ 
  3 & Day & 15.02 & 1.00 & 31.00 & 75.07 & 8.66 \\ 
  4 & Sold & 10.45 & 0.00 & 80995.00 & 24115.74 & 155.29 \\ 
  5 & Returned & 0.89 & 0.00 & 80995.00 & 23424.97 & 153.05 \\ 
   \hline
\end{tabular}
\caption{Summary of main variables} 
\end{table}

 
 As you can see in the table, it's a general overview around the dataset because meeting the exploratory information of the data you can assume you know the data set.


%SECCIÓN 3. PREGUNTAS INVESTIGACION
\section{Multivariate Analysis}
In this section you'll find the correlation analysis of variables to check the importance of them. This analysis allows checking the influence of an attribute on other, this analysis is commonly used to discard any attribute if it isn't important for other attributes.\\
\begin{knitrout}
\definecolor{shadecolor}{rgb}{0.969, 0.969, 0.969}\color{fgcolor}
\includegraphics[width=\maxwidth]{figure/pairs-1} 

\end{knitrout}

As you can see in the previous chart we can think some conclusions like these
\begin{itemize}
 \item Applying the Pearson's correlation, we can discard the value related with the Country, because the correlation index is too near than 0
 \item The year and the month are some of the most important variables of the data analysis of this study.
 \item The Sales affects directly the Devolutions
\end{itemize}


%SECCIÓN 3. PREGUNTAS INVESTIGACION
\section{Finding Patterns}
In this section we are going to establish some patterns between the data using the linear regression algorithm and the K-means algorithm.\\
\subsection{Linear Regression}
This algorithm is used to build a model to reproduce the information drawing the nearest line to all the points.\\
For this dataset we'll use this algorithm in the relationship between the month days and the quantity.
\begin{knitrout}
\definecolor{shadecolor}{rgb}{0.969, 0.969, 0.969}\color{fgcolor}
\includegraphics[width=\maxwidth]{figure/linear-regression-1} 

\end{knitrout}

\begin{knitrout}
\definecolor{shadecolor}{rgb}{0.969, 0.969, 0.969}\color{fgcolor}
\includegraphics[width=\maxwidth]{figure/linear-regression2-1} 

\end{knitrout}

\subsection{Clustering}
This algorithm is used to build a model to reproduce the information drawing the nearest centroid to all the points of the cluster.\\
For this dataset we'll use this algorithm in the relationship between the months days and the quantity.

\begin{knitrout}
\definecolor{shadecolor}{rgb}{0.969, 0.969, 0.969}\color{fgcolor}
\includegraphics[width=\maxwidth]{figure/kmeans-1} 

\includegraphics[width=\maxwidth]{figure/kmeans-2} 

\end{knitrout}




\section{Conclusions}
\begin{itemize}
 \item The data analysis of the dataset allows establishing some relationships and patterns to describe the behavior of the data.
 \item The critical task of the data analysis consists in understanding the dataset, the data preparation, and the tasks related with the patterns definition.
 \item The linear regression and the clustering using the K-means algorithm are good options to analyze some behaviors of the marketing.
 
 
\end{itemize}


%BIBLIOGRAFÍA

%ENTORNO {thebibliography}
%Permite al autor listar las referencias utilizadas y citarlas en algun punto del texto.

\begin{thebibliography}{1}
		
	\bibitem{biblio1} G. Canavos, Probabilidad y estadistica aplicaciones y metodos. Mc-Graw Hill, 1988. 
	\bibitem{biblio2} H. Chen, BUSINESS INTELLIGENCE AND ANALYTICS: FROM BIG DATA TO BIG IMPACT, MIS Quarterly Vol. 36 No. 4, pp. 1165-1188/December 2012 
	\bibitem{biblio3} Daqing Chen, Sai Liang Sain, and Kun Guo, Data mining for the online retail industry: A case study of RFM model-based customer segmentation using data mining, Journal of Database Marketing and Customer Strategy Management, Vol. 19, No. 3, pp. 197–208, 2012 
	\bibitem{biblio4} Ashishkumar Singh, Grace Rumantir, Annie South, Blair Bethwaite, Proceedings of the 2014 International Conference on Big Data Science and Computing.
	\bibitem{biblio5} A decision-making framework for precision marketing, Zhen You, Yain-Whar Si, Defu Zhang, XiangXiang Zeng, Stephen C.H. Leung c, Tao Li, Expert Systems with Applications, 42 (2015) 3357–3367.
	\bibitem{knitr-book} XIE, Yihui. Dynamic documents with R and knitr. Chapman \& Hall. Second edition. 2015.  
	\bibitem{latex-book} DE CASTRO KORGI, Rodrigo. El universo LATEX. Facultad de ciencias. Universidad Nacional de Colombia. Segunda edicion. 2003.  

\end{thebibliography}

\end{document}
