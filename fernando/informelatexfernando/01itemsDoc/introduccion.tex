%SECCIÓN 1. INTRODUCCIÓN 
\section{Introducción}
	El gobierno de Colombia, ha liberado datos de ámbito general donde terceros puedan tener acceso a ellos, los utilicen y generen servicios de perfil social, fortaleciendo así la transparencia de la información gubernamental.
	Las aplicaciones de los resultados de los análisis de los datos abiertos tienen como propósito facilitar la toma de decisiones con mayor certidumbre de forma eficiente.
	En el presente artículo analizaremos datos publicados por el DANE basados en una encuesta realizada en el 2015 donde se representan los resultados a preguntas relacionadas con los aportes al sistema de seguridad social colombiano, los cuales serán abordados en este estudio. El alcance de esta revisión se limitará a dos variables: a) edad, b) Monto de los aportes al sistema de seguridad social Colombiano.
	\\

\section{Datos Abiertos}

Según el Ministerio de Tecnologías de la Información y las Comunicaciones de la República de Colombia MinTIC, los Datos Abiertos son todos aquellos datos primarios, sin procesar, en formatos estándar, estructurados e interoperables que facilitan su acceso y permiten su reutilización, los cuales están bajo la custodia de las entidades públicas y que pueden ser obtenidos y ofrecidos sin reserva alguna, de forma libre y sin restricciones, con el fin que terceros puedan reutilizarlos y crear servicios derivados de los mismos.
\\
Los datos abiertos se caracterizan por lo siguiente:
  \begin{itemize}
  	\item \textbf{Disponibilidad y acceso}: la información debe estar disponible como un todo y a un costo razonable de reproducción, preferiblemente descargándola de internet. Además, la información debe estar disponible en una forma conveniente y modificable.
  	
	\item \textbf{Reutilización y redistribución}: los datos deben ser provistos bajo términos que permitan reutilizarlos y redistribuirlos, e incluso integrarlos con otros conjuntos de datos.
	
	\item \textbf{Participación universal}: todos deben poder utilizar, reutilizar y redistribuir la información. No debe haber discriminación alguna en términos de esfuerzo, personas o grupos. Restricciones “no comerciales” que prevendrían el uso comercial de los datos; o restricciones de uso para ciertos propósitos (por ejemplo sólo para educación) no son permitidos. (Public.Resource.Org, 2007).
	
  \end{itemize}

Los datos abiertos se representan de tal manera que permitan a los diversos países alcanzar sus metas tanto estratégicas como operativas pretendiendo “generar servicios de valor agregado a la sociedad, a través del desarrollo de aplicaciones realizadas por terceros (comunidades de desarrollo, industria infomediaria y academia), que utilizan los datos abiertos generados por las entidades de los diversos Estados”. 

El modelo de datos abiertos definido por el estado colombiano, tiene como principales componentes “los objetivos asociados a la implementación de los mismos en el país,  los principios de diseño en las  diferentes  perspectivas, el  modelo específico  con  los elementos  estratégicos,  tácticos,  operativos  y  de  soporte  y por último, el  mapa  de  ruta que  define  las  acciones  puntuales  a ejecutar para apropiar el modelo de datos abiertos definido”.

Una de las entidades del estado Colombiano que está a la vanguardia en el tema de datos abiertos es el Departamento Administrativo de Estadística DANE, el cual a la fecha ha puesto a disposición de público su información en formato de microdatos o Datos Enlazados, también conocidos como Linked Data LD, los cuales presentan la propiedad de vincularse de manera distribuida en la Web, de tal forma que se pueden referenciar al igual que lo hacen los enlaces de las páginas web (W3C.es, 2012).

Si bien el objeto de este artículo son el análisis de los datos abiertos expuestos por el DANE, actualmente este tema ha evolucionado en Datos Abiertos Enlazados, también conocidos como Linked Open Data LOD, que consisten en la simbiosis de los Datos Abiertos (Open Data) y los Datos Enlazados (Linked Data). 

El consorcio que define los estándares para la web, conocido como W3C, ha publicado en su glosario que los datos abiertos enlazados son un conjunto de datos que se encuentran asociados y disponibles en la Web pública, al igual que cuentan con licenciamiento abierto para su reutilización. La publicación de los datos abiertos enlazados en la web permite realizar consultas distribuidas sobre los conjuntos de datos, (2015e).

A partir del concepto de Datos Abiertos, entre el 7 y 8 de diciembre de 2007 se reunieron 30 defensores de gobierno abierto para desarrollar un conjunto de principios de los Datos de Gobierno Abierto, los cuales son también conocidos como Open Government Data OGD. Dicha reunión, celebrada en Sebastopol, California, fue diseñada para desarrollar una comprensión más sólida de por qué los datos del gobierno abierto son esenciales para la democracia (Public.Resource.Org, 2007).

La publicación de datos abiertos gubernamentales en internet ha sido definida por el Consorcio de la W3C (2009), en tres pasos sencillos:

  \begin{itemize}
  
  	\item Paso 1: publicación de los datos en su forma cruda. Los datos deben estar bien estructurados, lo cual permite que otros hagan uso automatizado de los datos con éxito, en formatos o estructuras conocidas (v.g. XML, RDF y CSV) facilitando ver los datos, en lugar de extraerlos (por ejemplo, imágenes de los datos), lo que debe evitarse por no ser útil (ibid).

	\item Paso 2: Crear un catálogo en línea de los datos en bruto (completo y con documentación) para que las personas puedan descubrir lo que se ha publicado. Estos conjuntos de datos en bruto deben ser estructurados y documentados de forma fiable, de lo contrario su utilidad es insignificante.

	\item Paso 3: Los datos han de ser legibles tanto por los humanos como por las máquinas de la siguiente manera:

		\begin{itemize}
			\item Codificar los datos utilizando estándares abiertos y de la industria o crear sus propias normas en base a su vocabulario.
			\item Que sus datos legibles puedan convertirse por cualquier persona o mediante el uso de transformaciones por software en tiempo real siguiendo los requisitos de accesibilidad.
			\item Utilizar de manera permanente identificadores normalizados y detectables.
			\item Permitir citaciones electrónicas en forma de hipervínculos estandarizados
		\end{itemize}
	\end{itemize}	

Estos pasos le ayudarán al público a encontrar, usar, citar y comprender los datos fácilmente. El catálogo de datos debe explicar las reglas o reglamentos que deben seguirse en el uso del conjunto de datos. Además, el catálogo de datos en sí se consideran "datos y debe ser publicado como datos estructurados, por lo que terceras personas pueden extraerlos sobre los dichos conjuntos.
