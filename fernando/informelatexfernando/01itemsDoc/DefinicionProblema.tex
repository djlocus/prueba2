%SECCIÓN 3. PREGUNTAS INVESTIGACION


\section{Preguntas de investigación}
Las preguntas de investigación permitirán analizar el comportamiento de los datos basados en el dataset, ya que a través de ellas se logra una mejor interpretación  y definición del problema.  Las preguntas de investigación se clasifican en varios tipos de acuerdo al análisis que se desea lograr y en este caso se van a desarrollar las siguientes: Nota: Utilizando K-means se  categorizaron las datos. Ver pág. 8.
 \subsection{Preguntas de carácter descriptivo}
 Las preguntas de carácter descriptivo permiten identificar y conocer las características iniciales del conjunto de datos. Las preguntas de carácter descriptivo son:
  \begin{itemize}
  \item ¿Cuál es la Media de los montos de los aportes de los ciudadanos cuya edad está entre los 22 a los 26 años, 27 a los 50 años, 51 a los 71 años, y 72 a los 94 años?
  \item ¿Cuál es el mayor y menor monto aportando al Sistema de Seguridad Social con edad superior a 72 años?
  \item ¿Cuál es el mayor y menor monto aportando al Sistema de Seguridad Social con edad superior a 51 años?
  \item ¿Cuál es el mayor y menor monto aportando al Sistema de Seguridad Social con edad superior a 27 años?
  \item ¿Cuál es el mayor y menor monto aportando al Sistema de Seguridad Social con edad superior a 22 años? 

  \end{itemize}
  \subsection{Preguntas de caracter exploratorio}
   Las preguntas de carácter exploratorio consisten en la búsqueda de patrones o relaciones que soporten una pregunta de investigación.
  \begin{itemize}
   \item ¿El valor máximo de los aportes realizados por la población colombiana al sistema de seguridad social aumentan con relación a la variable Edad?


  \end{itemize}
  \subsection{Preguntas de caracter inferencial}
   Las preguntas de carácter inferencial consisten en el planteamiento de una hipótesis que podría ser resuelta con el análisis respectivo de la información
  \begin{itemize}
   \item ¿Los colombianos que más aportan al sistema son ciudadanos que tienen una edad mayor a 30 años?
  \end{itemize}
