%SECCIÓN 4. Resultados
\section{Conclusiones}

En este papel se confirmó que el monto del valor aportado al Sistema de Seguridad Social Colombiano aumentaba con relación a la variable edad. De acuerdo con los resultados, después de los 50 años, los monos de los aportes superan los 140.000 pesos colombiano, cifra que corresponde a la media de los aportes sistema de seguridad social. También de induce de los resultados que los ciudadanos mayores que son pensionados o próximos a lograr su pensión realizan aportes por encima de la media es decir 140.000 pesos colombianos.  Con los resultados obtenidos se confirma la pregunta de investigación asociada con el monto de los aportes de los afiliados al sistema de seguridad social y su relación con la variable edad del cotizante. 

Con este análisis, también se confirma que los datos abiertos publicados por entes gubernamentales contribuyen a generar servicios con valor social, en razón a que el uso de este tipo de información permite la toma de decisiones soportados en datos que cumplen con las características de los datos abiertos. 

De otro lado, con los resultados de este papel, se concluye que los datos expuestos por el DANE, el análisis realizado y las técnicas de minería de datos permitieron confirmar la calidad de la información generada por esta entidad gubernamental del estado Colombiano. 

Finalmente, es importante resaltar que se logró brindar un resultados producto de un proceso de análisis basados en el marco metodológico expuesto por área de conocimiento con es la minería de datos, la cual brindó las herramientas para el procesos de validación de la calidad de los datos y la verificación y confirmación de la pregunta de investigación


\section{Trabajos futuros} 


Este trabajo permite definir trabajos adicionales que brinde certeza de la sostenibilidad del sistema y como el estado colombiano debe tomar acción para garantizar la sostenibilidad de este modelo de seguridad social Colombiano. 