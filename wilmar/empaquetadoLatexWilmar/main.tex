%Tipo de Documento [Conferencia]

\documentclass[conference]{IEEEtran}\usepackage[]{graphicx}\usepackage[]{color}
%% maxwidth is the original width if it is less than linewidth
%% otherwise use linewidth (to make sure the graphics do not exceed the margin)
\makeatletter
\def\maxwidth{ %
  \ifdim\Gin@nat@width>\linewidth
    \linewidth
  \else
    \Gin@nat@width
  \fi
}
\makeatother

\definecolor{fgcolor}{rgb}{0.345, 0.345, 0.345}
\newcommand{\hlnum}[1]{\textcolor[rgb]{0.686,0.059,0.569}{#1}}%
\newcommand{\hlstr}[1]{\textcolor[rgb]{0.192,0.494,0.8}{#1}}%
\newcommand{\hlcom}[1]{\textcolor[rgb]{0.678,0.584,0.686}{\textit{#1}}}%
\newcommand{\hlopt}[1]{\textcolor[rgb]{0,0,0}{#1}}%
\newcommand{\hlstd}[1]{\textcolor[rgb]{0.345,0.345,0.345}{#1}}%
\newcommand{\hlkwa}[1]{\textcolor[rgb]{0.161,0.373,0.58}{\textbf{#1}}}%
\newcommand{\hlkwb}[1]{\textcolor[rgb]{0.69,0.353,0.396}{#1}}%
\newcommand{\hlkwc}[1]{\textcolor[rgb]{0.333,0.667,0.333}{#1}}%
\newcommand{\hlkwd}[1]{\textcolor[rgb]{0.737,0.353,0.396}{\textbf{#1}}}%

\usepackage{framed}
\makeatletter
\newenvironment{kframe}{%
 \def\at@end@of@kframe{}%
 \ifinner\ifhmode%
  \def\at@end@of@kframe{\end{minipage}}%
  \begin{minipage}{\columnwidth}%
 \fi\fi%
 \def\FrameCommand##1{\hskip\@totalleftmargin \hskip-\fboxsep
 \colorbox{shadecolor}{##1}\hskip-\fboxsep
     % There is no \\@totalrightmargin, so:
     \hskip-\linewidth \hskip-\@totalleftmargin \hskip\columnwidth}%
 \MakeFramed {\advance\hsize-\width
   \@totalleftmargin\z@ \linewidth\hsize
   \@setminipage}}%
 {\par\unskip\endMakeFramed%
 \at@end@of@kframe}
\makeatother

\definecolor{shadecolor}{rgb}{.97, .97, .97}
\definecolor{messagecolor}{rgb}{0, 0, 0}
\definecolor{warningcolor}{rgb}{1, 0, 1}
\definecolor{errorcolor}{rgb}{1, 0, 0}
\newenvironment{knitrout}{}{} % an empty environment to be redefined in TeX

\usepackage{alltt}

%BIBLIOTECAS

% Este paquete se utiliza para generar texto o graficas de relleno.
%\usepackage{blindtext, graphicx}
%Biblioteca para graficas
\usepackage{graphicx}
%Biblioteca para lectura de caracteres ortográficos (tildes..etc. ) 
\usepackage[utf8]{inputenc}
%Biblioteca para enumeración de imagenes
\usepackage{float}
%Biblioteca para graficos vectrizados.svg 
\usepackage{svg}
\usepackage{enumerate}
%Biblioteca para enumerar figuras tablas.. etc en español 
\usepackage[spanish, es-tabla]{babel}
\usepackage[spanish]{babel}
%\usepackage[spanish,USenglish]{babel}
%%\usepackage{hyperref} %%%%%% para insertar hypervinculos
\usepackage[hidelinks]{hyperref}  %%para insertar hyperviculos sin cajas


%INICIO DEL DOCUMENTO
\IfFileExists{upquote.sty}{\usepackage{upquote}}{}
\begin{document}
	
	
	% TITULO DEL PAPER
\title{Análisis dinámico  a la Situación Actual de Desminado Humanitario en Colombia haciendo uso de la programación literaria y el paquete Knitr}

% NOMBRE DE LOS AUTORES
\author{
	\IEEEauthorblockN{Wilmar Diaz Rodriguez}
	\IEEEauthorblockA{Ingeniero de Electrónico\\ 
		Universidad Distrital Francisco José de Caldas\\
		Bogotá D.C., Colombia\\
		Email: wilmardiazr@gmail.com}
	%\and
}
%TITULO
\maketitle

%abstract del documento
	

%Iniciar Abstract
\begin{abstract}
It has been established that economic stratum is an aspect that determines the level of access to products and services that generate welfare in the life of a person, for this reason a person of high stratum compared with that of the lower stratum, it is difficult to obtaining welfare, beyond what can be termed non-core expenses.
\bigskip
The socioeconomic stratum is related to the amount of money that is accessed by a person to meet their material needs, but is there any relationship between socioeconomic strata and intellectual development? Is the choice to carry out professional studies at university public is given more to be a better option in cost reduction or rather academic recognition obtainable from it?
\end{abstract}

%Iniciar Palabras Clave Formato IEEE
\begin{IEEEkeywords}
	Minas, Data Mining, Desminado, Dataset, Modelo predictivo.
\end{IEEEkeywords}

%Introduccion 
%SECCIÓN 1. INTRODUCCIÓN 
\section{Introducción}
	El análisis de Desminado en Colombia es tema relevante para la actualidad ya que podría mostrar como al acercarse un acuerdo de Paz se ha aumentado o disminuido el Desminado en Colombia así como poder determinar cuales son los departamentos que mas se han visto afectados entre otros.


%Metodologia
%SECCIÓN 3. METODOLOGIA
\section{METODOLOGÍA}

 Para el desarrollo de este documento se utilizará como marco de referencia la metodología para minería de datos y fundamentos para análisis de datos [1].
 
  \subsection{Comprensión del negocio}
  
   \begin{itemize}
   	
   	\item \textbf{Identificar un problema}: El Gobierno Colombiano ha implementado un sistema de Seguridad Social el cual está compuesto por el régimen contributivo y subsidiado. En cuando al  régimen contributivo, se deben afiliar las personas que tienen una vinculación laboral, es decir, con capacidad de pago como los trabajadores formales e independientes, los pensionados y sus familias.  Debido al crecimiento poblacional Colombiano en los últimos años [2] es importante conocer el comportamiento las contribuciones a este sistema, y por consiguiente se hace necesario poder analizar el monto de los aportes al sistema de seguridad social según la edad del cotizante. Para ello, utilizaremos los datos abiertos publicados por el DANE.\\ 
   	
   \end{itemize} 
   
    \subsection{Preguntas de investigación}
    La pregunta de investigación que se desarrollará en este artículo estará enmarcada en los siguientes ámbitos:
    \begin{itemize}
    	\item Descriptivas 
    	\item Exploratorias
    	\item Inferenciales
    \end{itemize}
    
    Pregunta de Investigación: 
    \begin{itemize}
    	\item ¿El monto de los aportes de los afiliados al sistema de seguridad social colombiano aumenta con relación a la edad del cotizante?
    \end{itemize}
  
  \subsection{Comprensión de los Datos}
 
   \begin{itemize}
   \item \textbf{Recopilación inicial de datos}: Los datos obtenidos fueron exportados desde la página del Departamento Nacional de Estadísticas [2],  con el objetivo de analizar el monto de los aportes al sistema de seguridad social considerando la variable edad del cotizante.\\
   
   \item \textbf{Descripción de los datos: Variables del DataSet}: El conjunto de datos fueron descargados de la página del DANE en formato csv, formato separado por comas,, el contenía una población total de 10.000 registros.  el conjunto de datos está compuesto de los siguientes campos: 
   \\ 	 
   Género. Hombre - Mujer\\
   Año de Nacimiento\\
   Años cumplidos\\ 
   Valor mensual de la cotización a salud?\\
     
\end{itemize} 
  	
	
  \subsection{Preparación de los Datos}
  Siguiendo la metodología para minería de datos, se adelantarán las siguientes actividades 
  \begin{itemize}
  	\item Selección de los datos
  	\item Limpieza de datos
	\item Construcción de datos
    \item Integración de datos
	\item Formateo de datos
   \end{itemize}  
   Dado que la política de datos abiertos exigen que la disponibilidad de los datos al público, estén ya estructurados con antelación.  Dichas actividades no se llevan a cabo dentro de este artículo.

  \subsection{Análisis exploratorio}
Durante el análisis de la información de DANE se utilizan los siguientes aspectos para el desarrollo de este artículo:

  \begin{itemize}
  
   \item Frecuencia relativa    
   \item Medidas de tendencia central[4] 
	   \begin{itemize}
		 	\item Media. 
		 	\item Moda. 
		 	\item Mediana.
		 \end{itemize}
 %15\
 
  \item Medidas de Dispersión[4] 
  \begin{itemize}
 
  \item Varianza.  
 
 
     \item Desviaci\'on est\'andar .
\end{itemize} 

\end{itemize}  





%Preguntas de investigacion
%SECCIÓN 3. PREGUNTAS INVESTIGACION


\section{Preguntas de investigación}
Las preguntas de investigación juegan un papel importante para el desarrollo de una investigacion de esta indole, ya que a través de ellas se logra una mejor interpretación  y definición del problema.  Las preguntas de investigación se clasifican en varios tipos de acuerdo al análisis que se desea lograr y en este caso se van a desarrollar las siguientes:
 \subsection{Preguntas de carácter descriptivo}
 Cuando se responde las preguntas de carácter descriptivo ya se puede identificar y conocer las características iniciales del conjunto de datos. Las preguntas de carácter descriptivo son:
  \begin{itemize}
	 \item ¿La elección de realizar estudios profesionales en la universidad publica se da por que es una opción para reducción de costos?
	 \item ¿La elección de realizar estudios profesionales en la universidad publica se da por obtener reconocimiento académico de la misma?
  	
  \end{itemize}
  \subsection{Preguntas de carácter exploratorio}
   Las preguntas de carácter exploratorio consisten en la búsqueda de patrones o relaciones que soporten una pregunta de investigación.
   	  \begin{itemize}
  	  	\item ¿Existe algún tipo de relación entre el estrato socio económico y el desarrollo intelectual?
  	  	\item ¿Los estudiantes falsean el estrato social real, para obtener beneficios económicos en el valor de la matricula?
  	  	\item ¿La elección de estudiar en la Universidad de los Llanos, se realiza para reducir costos?
  	  	\item ¿La elección de estudiar en la Universidad de los Llanos, se realiza para obtener reconocimiento académico de la misma?
  	  \end{itemize}

  \subsection{Preguntas de carácter inferencial}
   Las preguntas de carácter inferencial consisten en el planteamiento de una hipótesis que podría ser resuelta con el análisis respectivo de la información
  \begin{itemize}
   \item    \item ¿Existe algún tipo de relación entre el estrato socio económico y el desarrollo intelectual?
  \end{itemize}
  \subsection{Preguntas de carácter predictivo}
   Las preguntas de carácter predictivo permiten analizar el comportamiento de la información a través del tiempo, con el objetivo de descubrir, proyectar, o realizar hipótesis sobre estados futuros.
	\begin{itemize}
		\item ¿Por definir?
	\end{itemize}
	
	% Analisis exploratorio

\section{Solución a preguntas}
El análisis dinámico de datos en la investigación reproducible debe permitir, reproducir el experimento propuesto para nuestro caso obtener el resultado de las preguntas de investigación de manera dinámica por tal razón de cambiar un dato en el dataset se va a ver reflejado en las respuestas del documento esto generalmente ocurrira mes a mes.
\\
\\
En este momento hay que tener varias consideraciones, se requiere hacer referencia a los datos anteponiendo la letra d ya que así esta construido el json que entrega la pagina de datos abiertos colombianos como ejemplo "json_data1\$d\$areadespejada[1]" esto mostraria el valor del area despejada para la primera fila del conjunto de datos notese que aparte de la variable donde se cargaron los datos se requiere hacer referencia a \$d,algunos de los datos que se van a tratar del datset son numéricos estos se deberán convertir a tal formato para poderlos tratar ya que cuando se lee dicho dataset se leen como texto para que R pueda trabajar con dichos datos se requiere la siguiente instrucción "json_data1\$d\$areadespejada <- as.numeric(json_data1\$d\$areadespejada)" lo que realiza esta instrucción es decirle que todos los datos de la columna area despejada los debe tratar como numéricos. 



Respuesta dinámica de las preguntas planteadas:
%\vspace{1mm}
\\
\\
En el siguiente apartado se muestran los datos estadisticos del área despejada así como el total de artefactos destruido.

% Table created by stargazer v.5.2 by Marek Hlavac, Harvard University. E-mail: hlavac at fas.harvard.edu
% Date and time: vie, jun 03, 2016 - 06:32:31 p.m.
\begin{table}[!htbp] \centering 
  \caption{Total de Artefactos y Área Despejada} 
  \label{} 
\begin{tabular}{@{\extracolsep{5pt}}lccccc} 
\\[-1.8ex]\hline 
\hline \\[-1.8ex] 
Statistic & \multicolumn{1}{c}{N} & \multicolumn{1}{c}{Mean} & \multicolumn{1}{c}{St. Dev.} & \multicolumn{1}{c}{Min} & \multicolumn{1}{c}{Max} \\ 
\hline \\[-1.8ex] 
totalartefactos & 530 & 9.872 & 44.424 & 0 & 476 \\ 
totalarea & 530 & 4,868.119 & 6,899.598 & 0 & 54,370 \\ 
\hline \\[-1.8ex] 
\end{tabular} 
\end{table} 



\begin{itemize}
	\item ¿Cuál es la Media de área despejada para los los años  2014, 2015, 2016?
\\
	Para poder manejar estos datos de forma dinámica lo que se debe realizar es una selección especifica de datos para este caso se realiza de la siguiente manera. vamos a verlo con el año 2016 datos2016<-subset(json_data1\$d,json_data1\$d\$ano==2016) de esta manera se seleccionan los datos que cumplan con el año y nuevamente se aplica los datos estadísticos con los siguientes resultados  


% Table created by stargazer v.5.2 by Marek Hlavac, Harvard University. E-mail: hlavac at fas.harvard.edu
% Date and time: vie, jun 03, 2016 - 06:32:32 p.m.
\begin{table}[!htbp] \centering 
  \caption{Total de Artefactos y Área Despejada para el año 2016} 
  \label{} 
\begin{tabular}{@{\extracolsep{5pt}}lccccc} 
\\[-1.8ex]\hline 
\hline \\[-1.8ex] 
Statistic & \multicolumn{1}{c}{N} & \multicolumn{1}{c}{Mean} & \multicolumn{1}{c}{St. Dev.} & \multicolumn{1}{c}{Min} & \multicolumn{1}{c}{Max} \\ 
\hline \\[-1.8ex] 
totalartefactos & 10 & 0.300 & 0.675 & 0 & 2 \\ 
totalarea & 10 & 978.600 & 1,359.855 & 0 & 4,179 \\ 
\hline \\[-1.8ex] 
\end{tabular} 
\end{table} 

	

% Table created by stargazer v.5.2 by Marek Hlavac, Harvard University. E-mail: hlavac at fas.harvard.edu
% Date and time: vie, jun 03, 2016 - 06:32:32 p.m.
\begin{table}[!htbp] \centering 
  \caption{Total de Artefactos y Área Despejada para el año 2015} 
  \label{} 
\begin{tabular}{@{\extracolsep{5pt}}lccccc} 
\\[-1.8ex]\hline 
\hline \\[-1.8ex] 
Statistic & \multicolumn{1}{c}{N} & \multicolumn{1}{c}{Mean} & \multicolumn{1}{c}{St. Dev.} & \multicolumn{1}{c}{Min} & \multicolumn{1}{c}{Max} \\ 
\hline \\[-1.8ex] 
totalartefactos & 97 & 2.330 & 5.554 & 0 & 37 \\ 
totalarea & 97 & 3,970.443 & 5,994.439 & 0 & 28,150 \\ 
\hline \\[-1.8ex] 
\end{tabular} 
\end{table} 



% Table created by stargazer v.5.2 by Marek Hlavac, Harvard University. E-mail: hlavac at fas.harvard.edu
% Date and time: vie, jun 03, 2016 - 06:32:32 p.m.
\begin{table}[!htbp] \centering 
  \caption{Total de Artefactos y Área Despejada para el año 2014} 
  \label{} 
\begin{tabular}{@{\extracolsep{5pt}}lccccc} 
\\[-1.8ex]\hline 
\hline \\[-1.8ex] 
Statistic & \multicolumn{1}{c}{N} & \multicolumn{1}{c}{Mean} & \multicolumn{1}{c}{St. Dev.} & \multicolumn{1}{c}{Min} & \multicolumn{1}{c}{Max} \\ 
\hline \\[-1.8ex] 
totalartefactos & 98 & 2.306 & 4.368 & 0 & 20 \\ 
totalarea & 98 & 5,612.071 & 9,010.254 & 0 & 43,194 \\ 
\hline \\[-1.8ex] 
\end{tabular} 
\end{table} 


		Para solucionar las preguntas que siguen a continuación se usaron los subconjuntos de datos para poder obtener los resultados requeridos de forma dinámica asi como algunas funciones propias de R

	
	\item ¿Qué departamentos tiene la mayor y menor área despejada de Minas en el año 2016?
	
Los departamentos que tiene la mayor área despejada para 2016 son:


[1] "ANTIOQUIA"

Los departamentos que tiene la menor área despejada para 2016 son:


[1] "ANTIOQUIA" "ANTIOQUIA" "SANTANDER" "SANTANDER" "TOLIMA"   




	\item ¿Qué departamentos tiene la mayor y menor área despejada de Minas en el año 2015?

	
	Los departamentos que tiene la mayor área despejada para 2015 son:
	
	
[1] "ANTIOQUIA"

	Los departamentos que tiene la menor área despejada para 2015 son:
	
	
 [1] "ANTIOQUIA" "ANTIOQUIA" "ANTIOQUIA" "ANTIOQUIA" "ANTIOQUIA"
 [6] "ANTIOQUIA" "ANTIOQUIA" "ANTIOQUIA" "ANTIOQUIA" "ANTIOQUIA"
[11] "ANTIOQUIA" "ANTIOQUIA" "ANTIOQUIA" "ANTIOQUIA" "ANTIOQUIA"
[16] "ANTIOQUIA" "ANTIOQUIA" "ANTIOQUIA" "ANTIOQUIA" "ANTIOQUIA"
[21] "ANTIOQUIA" "ANTIOQUIA" "ANTIOQUIA" "ANTIOQUIA" "ANTIOQUIA"
[26] "ANTIOQUIA" "ANTIOQUIA" "ANTIOQUIA" "ANTIOQUIA" "BOLIVAR"  
[31] "BOLIVAR"   "CALDAS"    "SANTANDER" "SANTANDER" "SANTANDER"
[36] "SANTANDER" "SANTANDER" "SANTANDER"




	\item ¿Qué departamentos tiene la mayor y menor área despejada de Minas en el año 2014?

	Los departamentos que tiene la mayor área despejada para 2014 son:
	
	
[1] "BOLIVAR"

	Los departamentos que tiene la menor área despejada para 2014 son:
	
	
 [1] "ANTIOQUIA" "ANTIOQUIA" "ANTIOQUIA" "ANTIOQUIA" "ANTIOQUIA"
 [6] "ANTIOQUIA" "ANTIOQUIA" "ANTIOQUIA" "ANTIOQUIA" "ANTIOQUIA"
[11] "ANTIOQUIA" "ANTIOQUIA" "ANTIOQUIA" "ANTIOQUIA" "ANTIOQUIA"
[16] "ANTIOQUIA" "ANTIOQUIA" "ANTIOQUIA" "ANTIOQUIA" "ANTIOQUIA"
[21] "ANTIOQUIA" "ANTIOQUIA" "ANTIOQUIA" "ANTIOQUIA" "ANTIOQUIA"
[26] "ANTIOQUIA" "ANTIOQUIA" "ANTIOQUIA" "ANTIOQUIA" "ANTIOQUIA"
[31] "BOLIVAR"   "BOLIVAR"   "BOLIVAR"   "CALDAS"    "CALDAS"   
[36] "META"      "SANTANDER" "SANTANDER" "SANTANDER" "SANTANDER"
[41] "SANTANDER" "SANTANDER"

	



	\item ¿Determinar si el acuerdo de paz ha permitido aumentar el área media despejada ?
	en conclusión no se ha visto aumento de desminado en colombia
	\item ¿Determinar si el acuerdo de paz ha permitido aumentar el área media despejada ?
	No
\end{itemize}

	
	% Solución de preguntas
%	<<child='codigoR/solucionQuestions.Rnw', echo=FALSE, results='hide'>>=  
%	@
	
	%BIBLIOGRAFÍA
	%ENTORNO {thebibliography}
	%Permite al autor listar las referencias utilizadas y citarlas en algun punto del texto.
	
	\newpage
	 \begin{thebibliography}{1}
	 	
	 	\bibitem{biblio1}
		Cornel University library The jsonlite Package: A Practical and Consistent Mapping Between JSON Data and R Objects
	 	\href{http://arxiv.org/abs/1403.2805}{\textcolor{blue}{\underline{URL1}}}

	 	\bibitem{biblio2}
	 	Situación de Desminado Humanitario en Colombia
	 	\href{http://www.datos.gov.co/frm/catalogo/frmCatalogo.aspx?dsId=75158}{\textcolor{blue}{\underline{URL2}}}
	 		 	
	 	\bibitem{biblio3}
		New package: jsonlite. A smart(er) JSON encoder/decoder.
		\href{https://www.opencpu.org/posts/jsonlite-a-smarter-json-encoder/}{\textcolor{blue}{\underline{URL3}}}
		
		\bibitem{biblio4}
		LINEAMIENTOS PARA LA IMPLEMENTACIÓN DE DATOS ABIERTOS EN COLOMBIA
		\href{http://ogdidata.blob.core.windows.net/documentos/articles-9407_GuiaDesarrolladores.pdf}{\textcolor{blue}{\underline{URL4}}}
		
		
	 \end{thebibliography}
	
\end{document}
