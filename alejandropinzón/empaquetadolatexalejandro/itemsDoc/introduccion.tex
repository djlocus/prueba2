%SECCIÓN 1. INTRODUCCIÓN 
\section{Introducción}

Se ha establecido que el estrato económico es un aspecto que determina el nivel de acceso a productos y servicios que generan bienestar en la vida de una persona, por esta razón una persona de estrato alto en comparación con aquella de estrato bajo, se le dificulta la obtención de su bienestar, mas allá de lo que se puede denominar gastos no básicos.

El estrato socio económico esta relacionado con la cantidad de dinero al que tiene acceso una persona para suplir sus necesidades materiales, pero ¿existe algún tipo de relación entre el estrato socio económico y el desarrollo intelectual? ¿La elección por llevar a cabo los estudios profesionales en la universidad publica se da más por ser una mejor opción en reducción de costos o más bien por el reconocimiento académico que puede obtenerse de la misma? 

Los estudiantes que conforman esta población se caracterizan por ser migrantes de otras regiones de la Orinoquia y la Amazonia, que se desplazan desde su sitio de origen con el fin de realizar estudios superiores de pregrado o postgrados. Una vez en la capital del Meta, Villavicencio su lugar de residencia varia desde la casa de un familiar, apartamentos arrendados o residencias estudiantiles.

Este trabajo tiene como fin determinar si la situación socio económica puede generar consecuencias directas sobre el rendimiento académico de los estudiantes de la Universidad de los Llanos, además de determinar que tan efectivo es la utilización de estrato socieconomico suministrado por los estudiantes al ingresar a la universidad.

Se sospecha de que las personas que ingresan a la universidad publica tienden a suministrar datos imprecisos respecto al estrato socieconomico al que pertenecen con el fin de obtener beneficios económicos en la matricula.

%	El análisis predictivo agrupa una variedad de técnicas estadísticas de modelización, aprendizaje automático y minería de datos que analiza los datos  actuales e históricos reales para hacer predicciones acerca del futuro o  acontecimientos no conocidos.\\
	%El análisis de datos siempre ha jugado un papel de vital importancia en la historia de la humanidad ya sea para comprender la naturaleza, mejorar la calidad de vida, el desarrollo de la economia, entre otras.\\
	%Además, la evolución de la tecnología ha representado un aumento considerable en cuanto a la capacidad de almacenamiento y procesamiento de información; lo cual permite el uso y tratamiento de grandes volumenes de datos.\\
	%La aplicacion del analisis de datos es infinita, puesto que todo aquello que puede ser clasificado y medido se puede analizar, por ejemplo el valor de la moneda frente a otros mercados, las visitas a un sitio web, el uso de alguna herramienta, la inteligencia de negocios, el analisis de ADN, etc.
	%Entre las diversas tecnicas para dicho analisis se pueden destacar la estadistica, el calculo de probabilidades, la mineria de datos, el big data, entre otros.\\
%	Por ello se utilizaron conceptos relacionados con estadistica y ``Bigdata'' para obtener informacion relevante sobre el conjunto de datos.
